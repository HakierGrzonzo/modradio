% !TEX program = xelatex
%Wzór dokumentu
%tu zmień marginesy i rozmiar czcionki
\documentclass[a4paper,12pt]{article}
\usepackage{inputenc}[utf8]
\usepackage[margin=2.8cm]{geometry}
\usepackage[polish]{babel}

%Lepiej tego nie zmieniaj, jak co to dodawaj pakiety
\usepackage{titlesec}
\usepackage{titling}
\usepackage{fancyhdr}
\usepackage{mdframed}
\usepackage{graphicx}
\usepackage{amsmath}
\usepackage{amsfonts}
\usepackage{multicol}
\usepackage{multirow}
\usepackage{listings}
\usepackage{caption}
\usepackage{float}
\usepackage{pdfpages}
\usepackage{tikz}
	\usetikzlibrary{arrows}
	\usetikzlibrary{patterns}
	\usetikzlibrary{decorations.pathmorphing}
\usepackage{pgf}
\usepackage[section]{placeins}
\usepackage{caption}



%inny wygląd
%\usepackage{tgbonum}


\usepackage{hyperref}
\hypersetup{
    colorlinks=true,
    linkcolor=blue,
    filecolor=magenta,      
    urlcolor=cyan,
}

\urlstyle{same}
%Zmienne, zmień je!
\graphicspath{ {./ilustracje/} }
\title{MODradio}
\author{Grzegorz Koperwas}
\date{\today}

%lokalizacja polska (odkomentuj jak piszesz po polsku)

\usepackage{polski}
\usepackage[polish]{babel} 
\usepackage{indentfirst}
\usepackage{icomma} 
\captionsetup[figure]{name=Załącznik}
\brokenpenalty=1000
\clubpenalty=1000
\widowpenalty=1000    

%nie odkometowuj wszystkiego, użyj mózgu
%\renewcommand\thechapter{\arabic{chapter}.}
\renewcommand\thesection{\arabic{section}.}
\renewcommand\thesubsection{\arabic{section}.\arabic{subsection}.}
\renewcommand\thesubsubsection{\arabic{subsubsection}.}

%Makra

\newcommand{\obrazek}[2]{
\begin{figure}[h]
    \centering
    \includegraphics[scale=#1]{#2}
\end{figure}
}     

\newcommand{\stopnie}{\ensuremath{^{\circ}}}

\newcommand{\twierdzonko}[1]{
    \begin{center}
    \begin{mdframed}
    #1
    \end{mdframed}          
    \end{center}
} 

\newcommand{\dwanajeden}[2]{
\ensuremath \left( \begin{array}{c}
    #1\\
    #2
\end{array} \right)
}  

%Stopka i head (sekcja której nie powinno się zmieniać)
\pagestyle{fancy}
\fancyhead{}
\fancyfoot{}

%Zmieniaj od tego miejsca
\rfoot{\thepage}
\lfoot{}
\lhead{}
\rhead{Ostatnia edycja: \today}
\renewcommand{\headrulewidth}{1pt}
\renewcommand{\footrulewidth}{1pt}



\begin{document}

\maketitle

\section{Temat projektu:}

Celem powstałego programu jest strumieniowanie piosenek z \texttt{trackerów}
poprzez \emph{Http Live Streaming\footnote{Dalej będę korzystał z skrótu
\textbf{HLS}}} w celu łatwego odsłuchu na urządzeniach
mobilnych.

Zatem musi on dynamicznie łączyć kolejne pliki w jeden strumień, bez
wcześniejszego wczytania ich wszystkich (Ilość tych plików wynosi 29 tysięcy,
rozmiar około 50 gigabajtów w formie skompresowanej).

\subsection*{Opis problemu:}

Archiwa strony \texttt{modarchive.org} są udostępniane w następującej formie:

\begin{enumerate}
        \item Piosenki znajdują się w drzewie folderów rozróżniającym je ze
            względu na format, artystę czy rok dodania. 

            Jakiekolwiek informacje zawarte w strukturze folderów mają być
            ignorowane, strumieniujemy piosenki w losowy sposób.

            Odtwarzacz VLC
            może uzyskiwać dostęp po protokole SMB do serwera z plikami, lecz
            nie odtwarza ich losowo w prosty sposób.

        \item Każda piosenka jest skompresowana jako archiwum \texttt{ZIP}.

        \item Piosenki są przechowywane jako \emph{moduły trackerów}, gdzie
            zamiast danych \texttt{PCM} przechowywane są w formie sampli i
            informacji jak je odtwarzać. Wynika to z architektury komputerów
            \emph{Amiga}, gdzie ten rodzaj muzyki powstał.
\end{enumerate}

Zatem program musi:

\begin{enumerate}
        \item Przyjmować pliki audio w formie pozwalającej na utrzymywanie
            \emph{bufora} następnych plików audio.
        \item Przyjmować pliki, lub ścieżki do nich z źródła łatwo dostępnego 
            dla jakiegoś skryptu. Na przykład przez \texttt{stdin}.
        \item Spełniać wymagania do łatwego zahostowania na moim klastrze
            \emph{Docker Swarm}, brak GUI itp.
\end{enumerate}

\section{Opis pobieranych danych przez program:}

Program pobiera przez standardowe wejście ścieżki do kolejnych plików audio.
Pliki te są \textbf{usuwane} po zakończeniu ich strumieniowania.

Program po napotkaniu pliku którego nie da się otworzyć lub zdekodować pomija
dany plik. 

Program stara się otwierać dwa pliki na raz. Plik, który jest właśnie
strumieniowany, oraz plik który jest następny w kolejce.

\section{Opis otrzymanych rezultatów}

\subsection*{Wydruk z programu}

Program wypisuje do konsoli logi diagnostyczne, za przechowywanie ich w plikach
odpowiada \textbf{systemd} lub \textbf{docker}.

Logi w omawianym przykładzie składają się z paru części:

\begin{itemize}
        \item Logi z znakami < oraz >, - Logi informujące jakie obiekty są
            tworzone przez program. Przykładowo:
            \begin{itemize}
                \item \texttt{<Encoder for aac>} - Stworzono obiekt kodera
                    formatu \texttt{AAC}, zawsze wyświetla się na początku
                    programu.
                \item \texttt{<Reader for /path/to/file>} - Stworzono obiekt
                    demuxera, który czyta zawartość pliku. Jeżeli dany plik
                    zawiera parę strumieni wideo lub audio, program wybiera
                    pierwszy strumień audio. Powinien wspierać większość
                    standardów (pewnie nawet zasoby sieciowe, zależy od wersji
                    biblioteki \texttt{libav}).
                \item \texttt{<Decoder for \$codec>} - Stworzono obiekt dekodera
                    kompresji, program \emph{powinien} wspierać wiele różnych
                    kodeków, jednak standard mp3 generuje nie poprawny dźwięk.
                \item \texttt{<Resampler from \$foo to \$bar>} - Stworzono
                    obiekt resamplera, który normalizuje częstotliwość
                    próbkowania oraz zapis bitowy.
            \end{itemize}
        \item Logi z znakami [ oraz ]\footnote{te kolorowe}, - Logi generowane przez bibliotekę
            \texttt{libav} - są zwykle w formie:

            \begin{center}
                [ \$źródło @ adres źródła ] \$wiadomość
            \end{center}
            
            Zwykle są to logi o statusie muxera \texttt{HLS}, lecz w przypadku
            złego pliku wejściowego zawierają one dodatkowe informacje o
            błędzie. Program \texttt{ffmpeg}, który jest frontendem do
            biblioteki \texttt{libav} generuje te same logi, więc jego
            dokumentacja pomoże w diagnozowaniu problemu.

        \item Reszta:

            Część logów w przykładzie pochodzi od skryptu realizującego
            przykładowe wykorzystanie programu. Jest on dołączony do kodu jako
            \texttt{./feeder.sh}.
\end{itemize}

Przykładowe logi znajdują się na załączniku \ref{lst:logs}.

\begin{figure}[h]
\begin{lstlisting}[frame=LB,basicstyle=\ttfamily\scriptsize, numbers=left]
[mpegts @ 0x7f0600355480] frame size not set
<Encoder for aac>
extracted /tmp/modfiles/lazertrack_heaven_2.mod
[aac @ 0x7f05f4005240] Estimating duration from bitrate, this may be inaccurate
<Reader for /tmp/modfiles/lazertrack_heaven_2.mod.aac>
<Decoder for aac>
<Resampler from 44100hz to 44100hz>
[hls @ 0x7f060034de00] Opening 'stream0.ts' for writing
[hls @ 0x7f060034de00] Opening 'stream.m3u8.tmp' for writing
extracted /tmp/modfiles/the_hardliner_-_whoronzon_gohonzon.xm
[aac @ 0x7f05ec001100] Estimating duration from bitrate, this may be inaccurate
<Reader for /tmp/modfiles/the_hardliner_-_whoronzon_gohonzon.xm.aac>
extracted /tmp/modfiles/the_savannus_never_never.669
[hls @ 0x7f060034de00] Opening 'stream1.ts' for writing
[hls @ 0x7f060034de00] Opening 'stream.m3u8.tmp' for writing
[hls @ 0x7f060034de00] Opening 'stream2.ts' for writing
[hls @ 0x7f060034de00] Opening 'stream.m3u8.tmp' for writing
extracted /tmp/modfiles/gustavo6046_-_trulix.it
extracted /tmp/modfiles/jabdah-cover.xm
extracted /tmp/modfiles/jason_ee-futurefuckballs2010_cover.it
extracted /tmp/modfiles/owcfullfrontal.it
[hls @ 0x7f060034de00] Opening 'stream3.ts' for writing
[hls @ 0x7f060034de00] Opening 'stream.m3u8.tmp' for writing
extracted /tmp/modfiles/skyline_-_boners.it
extracted /tmp/modfiles/ko0x_-_galaxy_guppy.it
[hls @ 0x7f060034de00] Opening 'stream4.ts' for writing
[hls @ 0x7f060034de00] Opening 'stream.m3u8.tmp' for writing
extracted /tmp/modfiles/pasyada_alex_-_decil.xm
extracted /tmp/modfiles/badboyremixhypnosis.mod
[hls @ 0x7f060034de00] Opening 'stream5.ts' for writing
[hls @ 0x7f060034de00] Opening 'stream.m3u8.tmp' for writing
Waiting for modradio to pickup data
[hls @ 0x7f060034de00] Opening 'stream6.ts' for writing
[hls @ 0x7f060034de00] Opening 'stream.m3u8.tmp' for writing
[hls @ 0x7f060034de00] Opening 'stream7.ts' for writing
[hls @ 0x7f060034de00] Opening 'stream.m3u8.tmp' for writing
...
\end{lstlisting}
\centering
    \caption{Przykładowe logi z programu.}
    \label{lst:logs}
\end{figure}

\subsection*{Pliki tworzone przez program:}

Program tworzy w aktualnym katalogu strumień w formacie \texttt{HLS} jako pliki
\texttt{stream\$x.ts} oraz plik ,,spis'' \texttt{stream.m3u8}. Te pliki powinny
być hostowane przez serwer HTTP, jako pliki statyczne. Programy takie jak
\emph{VLC Media Player}, \emph{ffplay}, \emph{safari} czy przeglądarki
internetowe na systemie android odtworzą je bez problemu, nawet z dysku. Dla
przeglądarek na komputerach PC trzeba dostarczyć demuxer w
\emph{javascript'cie}, co nie jest przedmiotem tego projektu.


\section{Zastosowane algorytmy:}

Obieg danych jest przedstawiony na załączniku \ref{rys:pipeline}.

W programie został zastosowany mechanizm asynchroniczności poprzez klasę
standardową \texttt{Future} oraz \texttt{ReaderAsyncFactory}.

\begin{figure}
    \resizebox{.3\textwidth}{!}{%
        
\begin{tikzpicture}[>=latex,line join=bevel,]
  \pgfsetlinewidth{1bp}
%%
\pgfsetcolor{black}
  % Edge: ReaderAsyncFactory -> Reader
  \draw [->] (56.5bp,485.58bp) .. controls (56.5bp,470.38bp) and (56.5bp,448.07bp)  .. (56.5bp,420.22bp);
  \definecolor{strokecol}{rgb}{0.0,0.0,0.0};
  \pgfsetstrokecolor{strokecol}
  \draw (81.0bp,456.8bp) node {Nazwa};
  \draw (81.0bp,441.8bp) node {pliku};
  % Edge: Reader -> Decoder
  \draw [->] (56.5bp,383.58bp) .. controls (56.5bp,368.38bp) and (56.5bp,346.07bp)  .. (56.5bp,318.22bp);
  \draw (115.5bp,354.8bp) node {Skompresowany};
  \draw (115.5bp,339.8bp) node {strumień};
  % Edge: Decoder -> Resampler
  \draw [->] (56.5bp,281.61bp) .. controls (56.5bp,269.24bp) and (56.5bp,252.37bp)  .. (56.5bp,228.05bp);
  \draw (87.0bp,255.0bp) node {Dane PCM};
  % Edge: Resampler -> Encoder
  \draw [->] (56.5bp,191.61bp) .. controls (56.5bp,179.24bp) and (56.5bp,162.37bp)  .. (56.5bp,138.05bp);
  \draw (87.0bp,165.0bp) node {Dane PCM};
  % Edge: Encoder -> Muxer
  \draw [->] (56.5bp,101.58bp) .. controls (56.5bp,86.382bp) and (56.5bp,64.067bp)  .. (56.5bp,36.219bp);
  \draw (115.5bp,72.8bp) node {Skompresowany};
  \draw (115.5bp,57.8bp) node {strumień};
  % Edge: stdin -> ReaderAsyncFactory
  \draw [->] (56.5bp,558.81bp) .. controls (56.5bp,550.79bp) and (56.5bp,541.05bp)  .. (56.5bp,522.03bp);
  % Node: ReaderAsyncFactory
\begin{scope}
  \definecolor{strokecol}{rgb}{0.0,0.0,0.0};
  \pgfsetstrokecolor{strokecol}
  \draw (56.5bp,504.0bp) ellipse (56.5bp and 18.0bp);
  \draw (56.5bp,504.0bp) node {ReaderAsyncFactory};
\end{scope}
  % Node: Reader
\begin{scope}
  \definecolor{strokecol}{rgb}{0.0,0.0,0.0};
  \pgfsetstrokecolor{strokecol}
  \draw (56.5bp,402.0bp) ellipse (27.0bp and 18.0bp);
  \draw (56.5bp,402.0bp) node {Reader};
\end{scope}
  % Node: Decoder
\begin{scope}
  \definecolor{strokecol}{rgb}{0.0,0.0,0.0};
  \pgfsetstrokecolor{strokecol}
  \draw (56.5bp,300.0bp) ellipse (29.0bp and 18.0bp);
  \draw (56.5bp,300.0bp) node {Decoder};
\end{scope}
  % Node: Resampler
\begin{scope}
  \definecolor{strokecol}{rgb}{0.0,0.0,0.0};
  \pgfsetstrokecolor{strokecol}
  \draw (56.5bp,210.0bp) ellipse (34.0bp and 18.0bp);
  \draw (56.5bp,210.0bp) node {Resampler};
\end{scope}
  % Node: Encoder
\begin{scope}
  \definecolor{strokecol}{rgb}{0.0,0.0,0.0};
  \pgfsetstrokecolor{strokecol}
  \draw (56.5bp,120.0bp) ellipse (29.0bp and 18.0bp);
  \draw (56.5bp,120.0bp) node {Encoder};
\end{scope}
  % Node: Muxer
\begin{scope}
  \definecolor{strokecol}{rgb}{0.0,0.0,0.0};
  \pgfsetstrokecolor{strokecol}
  \draw (56.5bp,18.0bp) ellipse (27.0bp and 18.0bp);
  \draw (56.5bp,18.0bp) node {Muxer};
\end{scope}
  % Node: stdin
\begin{scope}
  \definecolor{strokecol}{rgb}{0.0,0.0,0.0};
  \pgfsetstrokecolor{strokecol}
  \draw (56.5bp,577.0bp) ellipse (27.0bp and 18.0bp);
  \draw (56.5bp,577.0bp) node {stdin};
\end{scope}
%
\end{tikzpicture}

%
    }
    \centering
    \caption{Pipeline danych}
    \label{rys:pipeline}
\end{figure}

\section{Testy na poprawność działania programu:}

Poprawność działania była sprawdzana poprzez odsłuch strumieni generowanych
przez program oraz hostowanie ich serwerem \texttt{http} z biblioteki
standardowej języka \emph{Python}, poprzez komendę \texttt{python -m
http.server}.

\section{Wnioski:}

Po stworzeniu tego programu lepiej rozumiem architekturę biblioteki
\texttt{libav}, gdyż wcześniej wykorzystywałem ją tylko do dekodowania audio w
poprzednim projekcie z
\href{https://github.com/HakierGrzonzo/GrzesSFMLlib}{programowania II.} w języku
C++ oraz w projekcie komercyjnym dla Gliwickiej firmy \emph{APA
Group\footnote{Plakat rekrutacyjny ,,Pracuj w Czarnym Domu'' wisi na czwartym
piętrze wydziału Matematyki Stosowanej obok schodów.}} w języku
python.

\end{document}
