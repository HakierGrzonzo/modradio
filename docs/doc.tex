% !TEX program = xelatex
%Wzór dokumentu
%tu zmień marginesy i rozmiar czcionki
\documentclass[a4paper,12pt]{article}
\usepackage{inputenc}[utf8]
\usepackage[margin=2.8cm]{geometry}
\usepackage[polish]{babel}

%Lepiej tego nie zmieniaj, jak co to dodawaj pakiety
\usepackage{titlesec}
\usepackage{titling}
\usepackage{fancyhdr}
\usepackage{mdframed}
\usepackage{graphicx}
\usepackage{amsmath}
\usepackage{amsfonts}
\usepackage{multicol}
\usepackage{multirow}
\usepackage{listings}
\usepackage{caption}
\usepackage{float}
\usepackage{pdfpages}
\usepackage[x11names, svgnames, rgb]{xcolor}
\usepackage{tikz}
\usetikzlibrary{snakes,arrows,shapes}
\usepackage[section]{placeins}
\usepackage{caption}
%inny wygląd
%\usepackage{tgbonum}


\usepackage{hyperref}
\hypersetup{
    colorlinks=true,
    linkcolor=blue,
    filecolor=magenta,      
    urlcolor=cyan,
}

\urlstyle{same}
%Zmienne, zmień je!
\graphicspath{ {./ilustracje/} }
\title{MODradio - dokumentacja projektu}
\author{Grzegorz Koperwas}
\date{\today}

%lokalizacja polska (odkomentuj jak piszesz po polsku)

\usepackage{polski}
\usepackage[polish]{babel} 
\usepackage{indentfirst}
\usepackage{icomma} 
\captionsetup[figure]{name=Załącznik}
\brokenpenalty=1000
\clubpenalty=1000
\widowpenalty=1000    

%nie odkometowuj wszystkiego, użyj mózgu
%\renewcommand\thechapter{\arabic{chapter}.}
\renewcommand\thesection{\arabic{section}.}
\renewcommand\thesubsection{\arabic{section}.\arabic{subsection}.}
\renewcommand\thesubsubsection{\arabic{subsubsection}.}

%Makra

\renewcommand{\maketitle}{
    \begin{titlepage}
        \vspace*{2cm}
        
        \hspace{-3.4cm}\begin{minipage}{13cm}
            \hrule

            \vspace*{10pt}

            \hspace{15mm}\Large{\textbf{\thetitle}}

            \vspace*{5mm}

            \hspace{15mm}\large{\theauthor}

            \vspace*{10pt}

            \hrule
        \end{minipage}


        \vfill
    
        \begin{center}
            \footnotesize 
            Wydział Matematyki Stosowanej, Politechnika Śląska 
            
            \vspace*{1em}

            \today, Gliwice
        \end{center}
            
        \vspace*{1cm}
        
    \end{titlepage}
}

\newcommand{\obrazek}[2]{
\begin{figure}[h]
    \centering
    \includegraphics[scale=#1]{#2}
\end{figure}
}     

\newcommand{\stopnie}{\ensuremath{^{\circ}}}

\newcommand{\twierdzonko}[1]{
    \begin{center}
    \begin{mdframed}
    #1
    \end{mdframed}          
    \end{center}
} 

\newcommand{\dwanajeden}[2]{
\ensuremath \left( \begin{array}{c}
    #1\\
    #2
\end{array} \right)
}  

%Stopka i head (sekcja której nie powinno się zmieniać)
\pagestyle{fancy}
\fancyhead{}
\fancyfoot{}

%Zmieniaj od tego miejsca
\rfoot{\thepage}
\lfoot{}
\lhead{}
\rhead{Ostatnia edycja: \today}
\renewcommand{\headrulewidth}{1pt}
\renewcommand{\footrulewidth}{1pt}



\begin{document}

\maketitle

\section{Temat projektu:}

Celem powstałego programu jest strumieniowanie piosenek z \texttt{trackerów}
poprzez \emph{Http Live Streaming\footnote{Dalej będę korzystał ze skrótu
\textbf{HLS}}} w celu łatwego odsłuchu na urządzeniach
mobilnych.

W tym celu musi on dynamicznie łączyć kolejne pliki w jeden strumień, bez
wcześniejszego wczytania ich wszystkich\footnote{Ilość tych plików wynosi 29 tysięcy,
rozmiar około 50 gigabajtów w formie skompresowanej} do pamięci.

\subsection*{Opis problemu:}

Archiwa strony \texttt{modarchive.org} są udostępniane w następującej formie:

\begin{enumerate}
        \item Piosenki znajdują się w drzewie folderów rozróżniającym je ze
            względu na format, artystę czy rok dodania. 

            Jakiekolwiek informacje zawarte w strukturze folderów mają być
            ignorowane, odtwarzamy piosenki w prawdziwie losowy sposób.

            Odtwarzacz VLC
            może uzyskiwać dostęp po protokole SMB do serwera z plikami, lecz
            nie odtwarza ich losowo.

        \item Każda piosenka jest skompresowana jako archiwum \texttt{ZIP}.

        \item Piosenki są przechowywane jako \emph{moduły trackerów}, gdzie
            zamiast danych \texttt{PCM} przechowywane są w formie pojedynczych
            sampli oraz informacji jak je odtwarzać. Wynika to z architektury
            komputerów \emph{Amiga}, gdzie ten rodzaj muzyki powstał.
\end{enumerate}

Zatem program musi:

\begin{enumerate}
        \item Przyjmować pliki audio w formie pozwalającej na utrzymywanie
            \emph{bufora} następnych plików audio.
        \item Przyjmować pliki, lub ścieżki do nich ze źródła łatwo dostępnego 
            dla jakiegoś skryptu. Na przykład przez \texttt{stdin}.
        \item Spełniać wymagania do łatwego zahostowania na moim klastrze
            \emph{Docker Swarm}, brak GUI itp.
\end{enumerate}

\section*{Ważne rzeczy na które warto zwrócić uwagę:}

\begin{enumerate}
    \item Działanie programu było weryfikowane w środowisku: 
        \begin{itemize}
            \item System zgodny z \texttt{POSIX}, tutaj \texttt{Arch Linux}
            \item \texttt{OpenJDK Runtime Environment (build 11.0.13+8)}
            \item \texttt{Apache Maven 3.8.4}
        \end{itemize}
    \item Program używa natywnych bibliotek, przez to lista wspieranych platform
        ogranicza się do poniższych kombinacji architektur i systemów
        operacyjnych: 
        \begin{itemize}
            \item Android (arm, arm64, x86, x86\_64)
            \item iOS (arm, x86\_64\footnote{Używane dla emulatora systemu iOS,
                same urządzenia wykorzystują tylko arm})
            \item linux (armhf, arm64, ppc64le, x86, x86\_64)
            \item macOS (arm64, x86\_64)
            \item Windows (x86, x86\_64)
        \end{itemize}
    \item Dokumentacja jest kompilowana korzystając z narzędzia
        \texttt{xelatex}, jednak diagramy są generowane skryptem
        \texttt{dot2tex}. Plik \texttt{./Makefile} zawiera reguły dla programu
        \texttt{make}, które kompilują dokumentację po wydaniu polecenia:
        \begin{center}
            \texttt{make docs/doc.pdf}.
        \end{center}
    \item Program budujemy do postaci wykonywalnego pliku \texttt{.jar}
        poleceniem:
        \begin{center}
            \texttt{mvn assembly:assembly}
        \end{center}
\end{enumerate}

\section{Opis pobieranych danych przez program:}

Program pobiera przez standardowe wejście ścieżki do kolejnych plików audio.
Pliki te są \textbf{usuwane} po zakończeniu ich strumieniowania.

Program po napotkaniu pliku, którego nie da się otworzyć lub zdekodować pomija
dany plik. 

Program stara się otwierać dwa pliki naraz. Plik, który jest właśnie
strumieniowany, oraz plik, który jest następny w kolejce.

Program wspiera wiele formatów wejściowych, ich dokładna lista zależy od flag
użytych do kompilacji biblioteki ffmpeg. Jeżeli plik wejściowy zawiera więcej
niż jeden strumień\footnote{Na przykład plik wideo będzie zawierał zwykle jedną
ścieżkę wideo, jedną lub więcej audio oraz nawet kilkanaście ścieżek napisów,
czasami nawet miniaturkę jako jednoklatkowy strumień MJPEG.}, to program wybiera
pierwszy strumień audio.

\section{Opis otrzymanych rezultatów}

\subsection*{Wydruk z programu}

Program wypisuje do konsoli logi diagnostyczne, za przechowywanie ich w plikach
odpowiada \textbf{systemd} lub \textbf{docker}.

Przykładowe logi znajdują się na załączniku \ref{lst:logs}. Logi w załączniku 
składają się z paru części:

\begin{itemize}
        \item Na samym początku, po uruchomieniu, program wyświetla informacje z
            jakimi flagami została skompilowana biblioteka \texttt{libav}
        \item Logi ze znakami < oraz >, - Logi informujące jakie obiekty są
            tworzone przez program. Przykładowo:
            \begin{itemize}
                \item \texttt{<Encoder for aac>} - Stworzono obiekt kodera
                    formatu \texttt{AAC}, zawsze wyświetla się na początku
                    programu.
                \item \texttt{<Reader for /path/to/file>} - Stworzono obiekt
                    demuxera, który czyta zawartość pliku. Jeżeli dany plik
                    zawiera parę strumieni wideo lub audio, program wybiera
                    pierwszy strumień audio. Powinien wspierać większość
                    formatów\footnote{Pewnie nawet zasoby sieciowe, zależy od wersji
                    biblioteki \texttt{libav}.}, w tym na przykład
                    \texttt{mp4}.
                \item \texttt{<Decoder for \$codec>} - Stworzono obiekt dekodera
                    kompresji, program wspiera wiele różnych kodeków. Dokładna
                    lista zależy od opcji użytych do skompilowania pakietu
                    \texttt{org.bytedeco:ffmpeg-platform}.
                \item \texttt{<Resampler from \$foo to \$bar>} - Stworzono
                    obiekt resamplera, który normalizuje częstotliwość
                    próbkowania oraz zapis bitowy.
            \end{itemize}
        \item Logi z znakami [ oraz ]\footnote{te kolorowe}, - Logi generowane przez bibliotekę
            \texttt{libav} - są zwykle w formie:

            \begin{center}
                [ \$źródło @ adres źródła ] \$wiadomość
            \end{center}
            
            Zwykle są to logi o statusie muxera \texttt{HLS}, lecz w przypadku
            złego pliku wejściowego zawierają one dodatkowe informacje o
            błędzie. Program \texttt{ffmpeg}, który jest frontendem do
            biblioteki \texttt{libav} generuje te same logi, więc jego
            dokumentacja pomoże w diagnozowaniu problemu.

        \item Reszta:

            Część logów w przykładzie pochodzi od skryptu realizującego
            przykładowe wykorzystanie programu. Jest on dołączony do kodu jako
            \texttt{./feeder.sh}.
\end{itemize}


\begin{figure}[t]
\begin{lstlisting}[frame=LB,basicstyle=\ttfamily\scriptsize, numbers=left]
Using ffmpeg compiled with the following flags:
	--prefix=.. --disable-iconv --disable-opencl --disable-sdl2 --disable-bzlib
	--disable-lzma --disable-linux-perf --enable-shared --enable-version3 --enable-runtime-cpudetect
	--enable-zlib --enable-libmp3lame --enable-libspeex --enable-libopencore-amrnb
	--enable-libopencore-amrwb --enable-libvo-amrwbenc --enable-openssl --enable-libopenh264
	--enable-libvpx --enable-libfreetype --enable-libopus --enable-libxml2 --enable-libsrt
	--enable-cuda --enable-cuvid --enable-nvenc --enable-pthreads --enable-libxcb
	--cc='gcc -m64' --extra-cflags='-I../include/ -I../include/libxml2' --extra-ldflags=-L../lib/
	--extra-libs='-lstdc++ -lpthread -ldl -lz -lm '
[mpegts @ 0x7f0600355480] frame size not set
<Encoder for aac>
extracted /tmp/modfiles/lazertrack_heaven_2.mod
[aac @ 0x7f05f4005240] Estimating duration from bitrate, this may be inaccurate
<Reader for /tmp/modfiles/lazertrack_heaven_2.mod.aac>
<Decoder for aac>
<Resampler from 44100hz to 44100hz>
[hls @ 0x7f060034de00] Opening 'stream0.ts' for writing
[hls @ 0x7f060034de00] Opening 'stream.m3u8.tmp' for writing
extracted /tmp/modfiles/the_hardliner_-_whoronzon_gohonzon.xm
[aac @ 0x7f05ec001100] Estimating duration from bitrate, this may be inaccurate
<Reader for /tmp/modfiles/the_hardliner_-_whoronzon_gohonzon.xm.aac>
extracted /tmp/modfiles/the_savannus_never_never.669
[hls @ 0x7f060034de00] Opening 'stream1.ts' for writing
[hls @ 0x7f060034de00] Opening 'stream.m3u8.tmp' for writing
[hls @ 0x7f060034de00] Opening 'stream2.ts' for writing
[hls @ 0x7f060034de00] Opening 'stream.m3u8.tmp' for writing
extracted /tmp/modfiles/gustavo6046_-_trulix.it
extracted /tmp/modfiles/jabdah-cover.xm
extracted /tmp/modfiles/jason_ee-futurefuckballs2010_cover.it
extracted /tmp/modfiles/owcfullfrontal.it
[hls @ 0x7f060034de00] Opening 'stream3.ts' for writing
[hls @ 0x7f060034de00] Opening 'stream.m3u8.tmp' for writing
extracted /tmp/modfiles/skyline_-_boners.it
extracted /tmp/modfiles/ko0x_-_galaxy_guppy.it
[hls @ 0x7f060034de00] Opening 'stream4.ts' for writing
[hls @ 0x7f060034de00] Opening 'stream.m3u8.tmp' for writing
extracted /tmp/modfiles/pasyada_alex_-_decil.xm
extracted /tmp/modfiles/badboyremixhypnosis.mod
[hls @ 0x7f060034de00] Opening 'stream5.ts' for writing
[hls @ 0x7f060034de00] Opening 'stream.m3u8.tmp' for writing
Waiting for modradio to pickup data
[hls @ 0x7f060034de00] Opening 'stream6.ts' for writing
[hls @ 0x7f060034de00] Opening 'stream.m3u8.tmp' for writing
[hls @ 0x7f060034de00] Opening 'stream7.ts' for writing
[hls @ 0x7f060034de00] Opening 'stream.m3u8.tmp' for writing
...
\end{lstlisting}
\centering
    \caption{Przykładowe logi z programu.}
    \label{lst:logs}
\end{figure}

\subsection*{Pliki tworzone przez program:}

Program tworzy w aktualnym katalogu strumień w formacie \texttt{HLS} jako pliki
\texttt{stream\$x.ts} oraz plik ,,spis'' \texttt{stream.m3u8}. Te pliki powinny
być hostowane przez serwer HTTP, jako pliki statyczne. Programy takie jak
\emph{VLC Media Player}, \emph{ffplay}, \emph{safari} czy przeglądarki
internetowe na systemie android odtworzą je bez problemu, nawet z dysku. Dla
przeglądarek na komputerach PC trzeba dostarczyć demuxer w
\emph{JavaScript'cie}, co nie jest przedmiotem tego projektu.


\section{Zastosowane algorytmy:}

Obieg danych jest przedstawiony na załączniku \ref{rys:pipeline}.

W programie został zastosowany mechanizm asynchroniczności poprzez klasę
standardową \texttt{Future} oraz \texttt{ReaderAsyncFactory}, proces ten jest
pokazany w załączniku \ref{rys:reader}. 

Ogólny schemat blokowy jest pokazany w załączniku \ref{rys:program}.

\begin{figure}[p]
    \resizebox{.3\textwidth}{!}{%
        \input{pipeline.dot.tex}%
    }
    \centering
    \caption{Pipeline danych}
    \label{rys:pipeline}
\end{figure}

\begin{figure}[t]
    \resizebox{.5\textwidth}{!}{%
        \input{reader.dot.tex}%
    }
    \centering
    \caption{Schemat blokowy działania klasy \texttt{ReaderAsyncFactory}}
    \label{rys:reader}
\end{figure}

\begin{figure}[p]
    \resizebox{.54\textwidth}{!}{%
        \input{program.dot.tex}%
    }
    \centering
    \caption{Ogólny schemat działania programu.}
    \label{rys:program}
\end{figure}

\subsection*{Opis poszczególnych klas:}

\subsubsection{\texttt{App}}

Klasa zawierająca główną logikę programu w metodzie \texttt{main}.

\subsubsection{\texttt{Reader}}

Klasa odpowiedzialna jest za: 
\begin{itemize}
    \item Otwieranie demuxera dla zadanego pliku. 
    \item Wybieranie strumienia audio w pliku.
    \item Zwracanie kolejnych pakietów skompresowanego strumienia.
\end{itemize}

Jest ona zrealizowana głównie jako obudowanie struktury \texttt{AVFormatContext}
z biblioteki \texttt{libav}.

\subsubsection{\texttt{ReaderAsyncFactory}}

Klasa implementuje \texttt{Callable<Reader>}. Z tego powodu bardziej zachowuje
się jako funkcja, której działanie znajduje się na diagramie w załączniku
\ref{rys:reader}. Zrealizowana jest w ten sposób by ewentualny proces
,,zgadywania'' czy dany plik czymś użytecznym nie wpływał na główny wątek.

\subsubsection{\texttt{Decoder}}

Klasa opakowuje obiekty klasy \texttt{Reader}, w celu zdekodowania
informacji. Odpowiedzialna jest za: 
\begin{itemize}
    \item Stworzenie kontekstu kodeka \texttt{AVCodecContext}.
    \item Otworzenie samego kodeka w bibliotece \texttt{libav}.
    \item Zwracanie kolejnych zdekodowanych klatek.
\end{itemize}

\subsubsection{\texttt{Encoder}}

Klasa zarządza procesem kodowania klatek. Odpowiedzialna jest za: 
\begin{itemize}
    \item Tworzenie kontekstu kodeka \texttt{AVCodecContext} z zadanymi
        parametrami: 
        \begin{itemize}
            \item Rodzaj kodeka (używany jest tylko AAC)
            \item Bitrate strumienia
            \item Format bitowy próbek, który musi być wspierany przez kodek.
            \item Częstotliwość próbkowania
            \item Układ kanałów audio
        \end{itemize}
    \item Przekazywanie klatek do zakodowania do kodeka.
    \item Odbieranie zakodowanych pakietów, oraz informowanie kiedy takowe się
        już skończyły.
\end{itemize}

\subsubsection{\texttt{Resampler}}

Klasa konwertuje różne parametry strumieni dekodowanych przez obiekty klasy
\texttt{Decoder}, tak by mogły one dzielić jeden obiekt klasy \texttt{Encoder}.
Odpowiada za: 
\begin{itemize}
    \item Inicjalizację kontekstu resamplera \texttt{SwrContext}, w oparciu o
        parametry wejściowego oraz wyjściowego \texttt{AVCodecContext}.
    \item Przepisywanie \texttt{pts}, \texttt{dts} oraz \texttt{pos} między klatkami.
\end{itemize}

\subsubsection{\texttt{Muxer}}

Klasa odpowiedzialna jest za: 
\begin{itemize}
    \item Otwieranie muxera dla zadanego pliku. 
    \item Tworzenie strumienia audio w oparciu o \texttt{AVCodecContext}
        \texttt{Encoder}'a.
    \item Zapisywanie kolejnych pakietów w czasie rzeczywistym.
\end{itemize}


\section{Testy na poprawność działania programu:}

Poprawność działania była sprawdzana poprzez odsłuch strumieni generowanych
przez program oraz hostowanie ich serwerem \texttt{http} z biblioteki
standardowej języka \emph{Python}, poprzez komendę \texttt{python -m
http.server}.

\section{Wnioski:}

\subsection*{Znane błędy}

Program generuje strumień w tempie leciutko zawyżonym. Podczas dłuższego odsłuch
może wystąpić konieczność przewinięcia do przodu strumienia, bo starsze części
mogą już nie być dostępne. Wynika to z niejasności dokumentacji \texttt{libav}
na temat tego, w jakim formacie są dane synchronizujące zawarte w klatkach
strumienia. Przykładowo pliki w formacie \texttt{mp3} oraz \texttt{aac}
mają kompletnie różne wartości. 

Jednym z możliwych rozwiązań tego problemu byłoby manualne obliczanie długości
poszczególnych klatek w resamplerze, na podstawie ilość próbek (sampli),
częstotliwość próbkowania oraz ilość kanałów.

Innym błędem, raczej związanym z poprzednim\footnote{lub flag kompilacyjnych
związanych z \texttt{libmp3lame} w bibliotece \texttt{libav}, kwestie patentowe
komplikują sprawę wsparcia tego kodowania.}, jest błąd przy odczycie plików w
formacie \texttt{mp3}, program otwiera je poprawnie, resampluje z 48000hz do
41000hz, koduje do mp3, ale efekt wyjściowy nie przypomina zbytnio pliku
wejściowego.

\subsection*{Inne:}

Po stworzeniu tego programu lepiej rozumiem architekturę biblioteki
\texttt{libav}, gdyż wcześniej wykorzystywałem ją tylko do dekodowania audio w
poprzednim projekcie z
\href{https://github.com/HakierGrzonzo/GrzesSFMLlib}{programowania II.} w języku
C++ oraz w projekcie komercyjnym jako skrypty wykorzystujące frontend
\texttt{ffmpeg}. 

Program można by było wzbogacić o opcje do konfiguracji formatu wyjściowego, co
by sprawiło że mógłby być on uniwersalnym \emph{klejem} do plików audio.
Dodatkowo rozszerzenie go o strumienie video oraz napisów mogło by powiększyć
możliwości klejenia. Takie zastosowanie wymagałoby zastosowania wsparcia
akceleracji sprzętowej, która wymagała by testowania do kartach graficznych
wielu producentów\footnote{Firma Nvidia nie pozwala na transkodowanie więcej niż
trzech strumieni na kartach konsumenckich, wymagana jest karta z rodziny
co najmniej Quadro.}. 

\end{document}
